% Options for packages loaded elsewhere
\PassOptionsToPackage{unicode}{hyperref}
\PassOptionsToPackage{hyphens}{url}
%
\documentclass[
]{article}
\usepackage{amsmath,amssymb}
\usepackage{lmodern}
\usepackage{ifxetex,ifluatex}
\ifnum 0\ifxetex 1\fi\ifluatex 1\fi=0 % if pdftex
  \usepackage[T1]{fontenc}
  \usepackage[utf8]{inputenc}
  \usepackage{textcomp} % provide euro and other symbols
\else % if luatex or xetex
  \usepackage{unicode-math}
  \defaultfontfeatures{Scale=MatchLowercase}
  \defaultfontfeatures[\rmfamily]{Ligatures=TeX,Scale=1}
\fi
% Use upquote if available, for straight quotes in verbatim environments
\IfFileExists{upquote.sty}{\usepackage{upquote}}{}
\IfFileExists{microtype.sty}{% use microtype if available
  \usepackage[]{microtype}
  \UseMicrotypeSet[protrusion]{basicmath} % disable protrusion for tt fonts
}{}
\makeatletter
\@ifundefined{KOMAClassName}{% if non-KOMA class
  \IfFileExists{parskip.sty}{%
    \usepackage{parskip}
  }{% else
    \setlength{\parindent}{0pt}
    \setlength{\parskip}{6pt plus 2pt minus 1pt}}
}{% if KOMA class
  \KOMAoptions{parskip=half}}
\makeatother
\usepackage{xcolor}
\IfFileExists{xurl.sty}{\usepackage{xurl}}{} % add URL line breaks if available
\IfFileExists{bookmark.sty}{\usepackage{bookmark}}{\usepackage{hyperref}}
\hypersetup{
  pdftitle={R Notebook},
  hidelinks,
  pdfcreator={LaTeX via pandoc}}
\urlstyle{same} % disable monospaced font for URLs
\usepackage[margin=1in]{geometry}
\usepackage{color}
\usepackage{fancyvrb}
\newcommand{\VerbBar}{|}
\newcommand{\VERB}{\Verb[commandchars=\\\{\}]}
\DefineVerbatimEnvironment{Highlighting}{Verbatim}{commandchars=\\\{\}}
% Add ',fontsize=\small' for more characters per line
\usepackage{framed}
\definecolor{shadecolor}{RGB}{248,248,248}
\newenvironment{Shaded}{\begin{snugshade}}{\end{snugshade}}
\newcommand{\AlertTok}[1]{\textcolor[rgb]{0.94,0.16,0.16}{#1}}
\newcommand{\AnnotationTok}[1]{\textcolor[rgb]{0.56,0.35,0.01}{\textbf{\textit{#1}}}}
\newcommand{\AttributeTok}[1]{\textcolor[rgb]{0.77,0.63,0.00}{#1}}
\newcommand{\BaseNTok}[1]{\textcolor[rgb]{0.00,0.00,0.81}{#1}}
\newcommand{\BuiltInTok}[1]{#1}
\newcommand{\CharTok}[1]{\textcolor[rgb]{0.31,0.60,0.02}{#1}}
\newcommand{\CommentTok}[1]{\textcolor[rgb]{0.56,0.35,0.01}{\textit{#1}}}
\newcommand{\CommentVarTok}[1]{\textcolor[rgb]{0.56,0.35,0.01}{\textbf{\textit{#1}}}}
\newcommand{\ConstantTok}[1]{\textcolor[rgb]{0.00,0.00,0.00}{#1}}
\newcommand{\ControlFlowTok}[1]{\textcolor[rgb]{0.13,0.29,0.53}{\textbf{#1}}}
\newcommand{\DataTypeTok}[1]{\textcolor[rgb]{0.13,0.29,0.53}{#1}}
\newcommand{\DecValTok}[1]{\textcolor[rgb]{0.00,0.00,0.81}{#1}}
\newcommand{\DocumentationTok}[1]{\textcolor[rgb]{0.56,0.35,0.01}{\textbf{\textit{#1}}}}
\newcommand{\ErrorTok}[1]{\textcolor[rgb]{0.64,0.00,0.00}{\textbf{#1}}}
\newcommand{\ExtensionTok}[1]{#1}
\newcommand{\FloatTok}[1]{\textcolor[rgb]{0.00,0.00,0.81}{#1}}
\newcommand{\FunctionTok}[1]{\textcolor[rgb]{0.00,0.00,0.00}{#1}}
\newcommand{\ImportTok}[1]{#1}
\newcommand{\InformationTok}[1]{\textcolor[rgb]{0.56,0.35,0.01}{\textbf{\textit{#1}}}}
\newcommand{\KeywordTok}[1]{\textcolor[rgb]{0.13,0.29,0.53}{\textbf{#1}}}
\newcommand{\NormalTok}[1]{#1}
\newcommand{\OperatorTok}[1]{\textcolor[rgb]{0.81,0.36,0.00}{\textbf{#1}}}
\newcommand{\OtherTok}[1]{\textcolor[rgb]{0.56,0.35,0.01}{#1}}
\newcommand{\PreprocessorTok}[1]{\textcolor[rgb]{0.56,0.35,0.01}{\textit{#1}}}
\newcommand{\RegionMarkerTok}[1]{#1}
\newcommand{\SpecialCharTok}[1]{\textcolor[rgb]{0.00,0.00,0.00}{#1}}
\newcommand{\SpecialStringTok}[1]{\textcolor[rgb]{0.31,0.60,0.02}{#1}}
\newcommand{\StringTok}[1]{\textcolor[rgb]{0.31,0.60,0.02}{#1}}
\newcommand{\VariableTok}[1]{\textcolor[rgb]{0.00,0.00,0.00}{#1}}
\newcommand{\VerbatimStringTok}[1]{\textcolor[rgb]{0.31,0.60,0.02}{#1}}
\newcommand{\WarningTok}[1]{\textcolor[rgb]{0.56,0.35,0.01}{\textbf{\textit{#1}}}}
\usepackage{graphicx}
\makeatletter
\def\maxwidth{\ifdim\Gin@nat@width>\linewidth\linewidth\else\Gin@nat@width\fi}
\def\maxheight{\ifdim\Gin@nat@height>\textheight\textheight\else\Gin@nat@height\fi}
\makeatother
% Scale images if necessary, so that they will not overflow the page
% margins by default, and it is still possible to overwrite the defaults
% using explicit options in \includegraphics[width, height, ...]{}
\setkeys{Gin}{width=\maxwidth,height=\maxheight,keepaspectratio}
% Set default figure placement to htbp
\makeatletter
\def\fps@figure{htbp}
\makeatother
\setlength{\emergencystretch}{3em} % prevent overfull lines
\providecommand{\tightlist}{%
  \setlength{\itemsep}{0pt}\setlength{\parskip}{0pt}}
\setcounter{secnumdepth}{-\maxdimen} % remove section numbering
\ifluatex
  \usepackage{selnolig}  % disable illegal ligatures
\fi

\title{R Notebook}
\author{}
\date{\vspace{-2.5em}}

\begin{document}
\maketitle

\begin{Shaded}
\begin{Highlighting}[]
\CommentTok{\#setwd("\textasciitilde{}/R\_KSU/ML/Assignment 3")}
\NormalTok{bank }\OtherTok{\textless{}{-}} \FunctionTok{read.csv}\NormalTok{(}\StringTok{"UniversalBank.csv"}\NormalTok{)}
\end{Highlighting}
\end{Shaded}

\begin{Shaded}
\begin{Highlighting}[]
\FunctionTok{library}\NormalTok{(reshape)}
\end{Highlighting}
\end{Shaded}

\begin{verbatim}
## Warning: package 'reshape' was built under R version 4.0.3
\end{verbatim}

\begin{Shaded}
\begin{Highlighting}[]
\FunctionTok{library}\NormalTok{(reshape2)}
\end{Highlighting}
\end{Shaded}

\begin{verbatim}
## Warning: package 'reshape2' was built under R version 4.0.3
\end{verbatim}

\begin{verbatim}
## 
## Attaching package: 'reshape2'
\end{verbatim}

\begin{verbatim}
## The following objects are masked from 'package:reshape':
## 
##     colsplit, melt, recast
\end{verbatim}

\begin{Shaded}
\begin{Highlighting}[]
\FunctionTok{str}\NormalTok{(bank)}
\end{Highlighting}
\end{Shaded}

\begin{verbatim}
## 'data.frame':    5000 obs. of  14 variables:
##  $ ID                : int  1 2 3 4 5 6 7 8 9 10 ...
##  $ Age               : int  25 45 39 35 35 37 53 50 35 34 ...
##  $ Experience        : int  1 19 15 9 8 13 27 24 10 9 ...
##  $ Income            : int  49 34 11 100 45 29 72 22 81 180 ...
##  $ ZIP.Code          : int  91107 90089 94720 94112 91330 92121 91711 93943 90089 93023 ...
##  $ Family            : int  4 3 1 1 4 4 2 1 3 1 ...
##  $ CCAvg             : num  1.6 1.5 1 2.7 1 0.4 1.5 0.3 0.6 8.9 ...
##  $ Education         : int  1 1 1 2 2 2 2 3 2 3 ...
##  $ Mortgage          : int  0 0 0 0 0 155 0 0 104 0 ...
##  $ Personal.Loan     : int  0 0 0 0 0 0 0 0 0 1 ...
##  $ Securities.Account: int  1 1 0 0 0 0 0 0 0 0 ...
##  $ CD.Account        : int  0 0 0 0 0 0 0 0 0 0 ...
##  $ Online            : int  0 0 0 0 0 1 1 0 1 0 ...
##  $ CreditCard        : int  0 0 0 0 1 0 0 1 0 0 ...
\end{verbatim}

\begin{Shaded}
\begin{Highlighting}[]
\FunctionTok{summary}\NormalTok{(bank)}
\end{Highlighting}
\end{Shaded}

\begin{verbatim}
##        ID            Age          Experience       Income          ZIP.Code    
##  Min.   :   1   Min.   :23.00   Min.   :-3.0   Min.   :  8.00   Min.   : 9307  
##  1st Qu.:1251   1st Qu.:35.00   1st Qu.:10.0   1st Qu.: 39.00   1st Qu.:91911  
##  Median :2500   Median :45.00   Median :20.0   Median : 64.00   Median :93437  
##  Mean   :2500   Mean   :45.34   Mean   :20.1   Mean   : 73.77   Mean   :93153  
##  3rd Qu.:3750   3rd Qu.:55.00   3rd Qu.:30.0   3rd Qu.: 98.00   3rd Qu.:94608  
##  Max.   :5000   Max.   :67.00   Max.   :43.0   Max.   :224.00   Max.   :96651  
##      Family          CCAvg          Education        Mortgage    
##  Min.   :1.000   Min.   : 0.000   Min.   :1.000   Min.   :  0.0  
##  1st Qu.:1.000   1st Qu.: 0.700   1st Qu.:1.000   1st Qu.:  0.0  
##  Median :2.000   Median : 1.500   Median :2.000   Median :  0.0  
##  Mean   :2.396   Mean   : 1.938   Mean   :1.881   Mean   : 56.5  
##  3rd Qu.:3.000   3rd Qu.: 2.500   3rd Qu.:3.000   3rd Qu.:101.0  
##  Max.   :4.000   Max.   :10.000   Max.   :3.000   Max.   :635.0  
##  Personal.Loan   Securities.Account   CD.Account         Online      
##  Min.   :0.000   Min.   :0.0000     Min.   :0.0000   Min.   :0.0000  
##  1st Qu.:0.000   1st Qu.:0.0000     1st Qu.:0.0000   1st Qu.:0.0000  
##  Median :0.000   Median :0.0000     Median :0.0000   Median :1.0000  
##  Mean   :0.096   Mean   :0.1044     Mean   :0.0604   Mean   :0.5968  
##  3rd Qu.:0.000   3rd Qu.:0.0000     3rd Qu.:0.0000   3rd Qu.:1.0000  
##  Max.   :1.000   Max.   :1.0000     Max.   :1.0000   Max.   :1.0000  
##    CreditCard   
##  Min.   :0.000  
##  1st Qu.:0.000  
##  Median :0.000  
##  Mean   :0.294  
##  3rd Qu.:1.000  
##  Max.   :1.000
\end{verbatim}

\begin{Shaded}
\begin{Highlighting}[]
\NormalTok{bank}\SpecialCharTok{$}\NormalTok{Personal.Loan }\OtherTok{=} \FunctionTok{as.factor}\NormalTok{(bank}\SpecialCharTok{$}\NormalTok{Personal.Loan)}
\NormalTok{bank}\SpecialCharTok{$}\NormalTok{Online }\OtherTok{=} \FunctionTok{as.factor}\NormalTok{(bank}\SpecialCharTok{$}\NormalTok{Online)}
\NormalTok{bank}\SpecialCharTok{$}\NormalTok{CreditCard }\OtherTok{=} \FunctionTok{as.factor}\NormalTok{(bank}\SpecialCharTok{$}\NormalTok{CreditCard)}
\end{Highlighting}
\end{Shaded}

\begin{Shaded}
\begin{Highlighting}[]
\FunctionTok{set.seed}\NormalTok{(}\DecValTok{1}\NormalTok{)}
\NormalTok{train.index }\OtherTok{\textless{}{-}} \FunctionTok{sample}\NormalTok{(}\FunctionTok{row.names}\NormalTok{(bank), }\FloatTok{0.7}\SpecialCharTok{*}\FunctionTok{dim}\NormalTok{(bank)[}\DecValTok{1}\NormalTok{])  }
\NormalTok{test.index }\OtherTok{\textless{}{-}} \FunctionTok{setdiff}\NormalTok{(}\FunctionTok{row.names}\NormalTok{(bank), train.index) }
\NormalTok{train }\OtherTok{\textless{}{-}}\NormalTok{ bank[train.index, ]}
\NormalTok{test }\OtherTok{\textless{}{-}}\NormalTok{ bank[test.index,]}
\end{Highlighting}
\end{Shaded}

A. Create a pivot table for the training data with Online as a column
variable, CC as a row variable, and Loan as a secondary row variable.
The values inside the table should convey the count. In R use functions
melt() and cast(), or function table(). In Python, use panda dataframe
methods melt() and pivot().

\begin{Shaded}
\begin{Highlighting}[]
\FunctionTok{table}\NormalTok{(}\StringTok{"CC"}\OtherTok{=}\NormalTok{bank}\SpecialCharTok{$}\NormalTok{CreditCard,}\StringTok{"PL"}\OtherTok{=}\NormalTok{bank}\SpecialCharTok{$}\NormalTok{Personal.Loan,}\StringTok{"O/L"}\OtherTok{=}\NormalTok{bank}\SpecialCharTok{$}\NormalTok{Online)}
\end{Highlighting}
\end{Shaded}

\begin{verbatim}
## , , O/L = 0
## 
##    PL
## CC     0    1
##   0 1300  128
##   1  527   61
## 
## , , O/L = 1
## 
##    PL
## CC     0    1
##   0 1893  209
##   1  800   82
\end{verbatim}

\begin{Shaded}
\begin{Highlighting}[]
\NormalTok{t1}\OtherTok{=} \FunctionTok{recast}\NormalTok{(bank,bank}\SpecialCharTok{$}\NormalTok{CreditCard}\SpecialCharTok{+}\NormalTok{bank}\SpecialCharTok{$}\NormalTok{Personal.Loan}\SpecialCharTok{\textasciitilde{}}\NormalTok{bank}\SpecialCharTok{$}\NormalTok{Online)}
\end{Highlighting}
\end{Shaded}

\begin{verbatim}
## Using Personal.Loan, Online, CreditCard as id variables
\end{verbatim}

\begin{verbatim}
## Aggregation function missing: defaulting to length
\end{verbatim}

\begin{Shaded}
\begin{Highlighting}[]
\NormalTok{t1}
\end{Highlighting}
\end{Shaded}

\begin{verbatim}
##   bank$CreditCard bank$Personal.Loan    0    1
## 1               0                  0 1300 1893
## 2               0                  1  128  209
## 3               1                  0  527  800
## 4               1                  1   61   82
\end{verbatim}

B. Consider the task of classifying a customer who owns a bank credit
card and is actively using online banking services. Looking at the pivot
table, what is the probability that this customer will accept the loan
offer? {[}This is the probability of loan acceptance (Loan = 1)
conditional on having a bank credit card (CC = 1) and being an active
user of online banking services (Online = 1){]}.

\begin{quote}
\begin{quote}
Probability of Loan acceptance given having a bank credit card and user
of online services is 82/882 = 0.09297
\end{quote}
\end{quote}

C. Create two separate pivot tables for the training data. One will have
Loan (rows) as a function of Online (columns) and the other will have
Loan (rows) as a function of CC.

\begin{Shaded}
\begin{Highlighting}[]
\NormalTok{t2}\OtherTok{=} \FunctionTok{recast}\NormalTok{(bank,bank}\SpecialCharTok{$}\NormalTok{Personal.Loan}\SpecialCharTok{\textasciitilde{}}\NormalTok{bank}\SpecialCharTok{$}\NormalTok{Online)}
\end{Highlighting}
\end{Shaded}

\begin{verbatim}
## Using Personal.Loan, Online, CreditCard as id variables
\end{verbatim}

\begin{verbatim}
## Aggregation function missing: defaulting to length
\end{verbatim}

\begin{Shaded}
\begin{Highlighting}[]
\NormalTok{t2}
\end{Highlighting}
\end{Shaded}

\begin{verbatim}
##   bank$Personal.Loan    0    1
## 1                  0 1827 2693
## 2                  1  189  291
\end{verbatim}

\begin{Shaded}
\begin{Highlighting}[]
\NormalTok{t3}\OtherTok{=} \FunctionTok{recast}\NormalTok{(bank,bank}\SpecialCharTok{$}\NormalTok{CreditCard}\SpecialCharTok{\textasciitilde{}}\NormalTok{bank}\SpecialCharTok{$}\NormalTok{Online)}
\end{Highlighting}
\end{Shaded}

\begin{verbatim}
## Using Personal.Loan, Online, CreditCard as id variables
\end{verbatim}

\begin{verbatim}
## Aggregation function missing: defaulting to length
\end{verbatim}

\begin{Shaded}
\begin{Highlighting}[]
\NormalTok{t3}
\end{Highlighting}
\end{Shaded}

\begin{verbatim}
##   bank$CreditCard    0    1
## 1               0 1428 2102
## 2               1  588  882
\end{verbatim}

D. Compute the following quantities {[}P(A \textbar{} B) means ``the
probability of A given B''{]}: i. P(CC = 1 \textbar{} Loan = 1) (the
proportion of credit card holders among the loan acceptors) ii. P(Online
= 1 \textbar{} Loan = 1) iii. P(Loan = 1) (the proportion of loan
acceptors) iv. P(CC = 1 \textbar{} Loan = 0) v. P(Online = 1 \textbar{}
Loan = 0) vi. P(Loan = 0)

\begin{Shaded}
\begin{Highlighting}[]
\FunctionTok{table}\NormalTok{(train[,}\FunctionTok{c}\NormalTok{(}\DecValTok{14}\NormalTok{,}\DecValTok{10}\NormalTok{)])}
\end{Highlighting}
\end{Shaded}

\begin{verbatim}
##           Personal.Loan
## CreditCard    0    1
##          0 2241  232
##          1  933   94
\end{verbatim}

\begin{Shaded}
\begin{Highlighting}[]
\FunctionTok{table}\NormalTok{(train[,}\FunctionTok{c}\NormalTok{(}\DecValTok{13}\NormalTok{,}\DecValTok{10}\NormalTok{)])}
\end{Highlighting}
\end{Shaded}

\begin{verbatim}
##       Personal.Loan
## Online    0    1
##      0 1304  132
##      1 1870  194
\end{verbatim}

\begin{Shaded}
\begin{Highlighting}[]
\FunctionTok{table}\NormalTok{(train[,}\FunctionTok{c}\NormalTok{(}\DecValTok{10}\NormalTok{)])}
\end{Highlighting}
\end{Shaded}

\begin{verbatim}
## 
##    0    1 
## 3174  326
\end{verbatim}

P(Cc\textbar Pl) = 94/(94+232) = 0.28834 P(Ol\textbar Pl) =
194/(194+132) = 0.59509 P(Pl) = 326/(326+3174) = 0.09314
P(Cc\textbar Pl')= 933/(933+2241) = 0.29395 P(Ol\textbar Pl')=
1870/(1870+1304)= 0.58916 P(Pl') = 3174/(3174+326) = 0.90685

E. Use the quantities computed above to compute the naive Bayes
probability P(Loan = 1 \textbar{} CC = 1, Online = 1).

\begin{Shaded}
\begin{Highlighting}[]
\NormalTok{((}\DecValTok{94}\SpecialCharTok{/}\NormalTok{(}\DecValTok{94}\SpecialCharTok{+}\DecValTok{232}\NormalTok{))}\SpecialCharTok{*}\NormalTok{(}\DecValTok{194}\SpecialCharTok{/}\NormalTok{(}\DecValTok{194}\SpecialCharTok{+}\DecValTok{132}\NormalTok{))}\SpecialCharTok{*}\NormalTok{(}\DecValTok{326}\SpecialCharTok{/}\NormalTok{(}\DecValTok{326}\SpecialCharTok{+}\DecValTok{3174}\NormalTok{)))}\SpecialCharTok{/}\NormalTok{(((}\DecValTok{94}\SpecialCharTok{/}\NormalTok{(}\DecValTok{94}\SpecialCharTok{+}\DecValTok{232}\NormalTok{))}\SpecialCharTok{*}\NormalTok{(}\DecValTok{194}\SpecialCharTok{/}\NormalTok{(}\DecValTok{194}\SpecialCharTok{+}\DecValTok{132}\NormalTok{))}\SpecialCharTok{*}\NormalTok{(}\DecValTok{326}\SpecialCharTok{/}\NormalTok{(}\DecValTok{326}\SpecialCharTok{+}\DecValTok{3174}\NormalTok{)))}\SpecialCharTok{+}\NormalTok{((}\DecValTok{933}\SpecialCharTok{/}\NormalTok{(}\DecValTok{933}\SpecialCharTok{+}\DecValTok{2241}\NormalTok{))}\SpecialCharTok{*}\NormalTok{(}\DecValTok{1870}\SpecialCharTok{/}\NormalTok{(}\DecValTok{1870}\SpecialCharTok{+}\DecValTok{1304}\NormalTok{))}\SpecialCharTok{*}\DecValTok{3174}\SpecialCharTok{/}\NormalTok{(}\DecValTok{3174}\SpecialCharTok{+}\DecValTok{326}\NormalTok{)))}
\end{Highlighting}
\end{Shaded}

\begin{verbatim}
## [1] 0.09236489
\end{verbatim}

F. Compare this value with the one obtained from the pivot table in (B).
Which is a more accurate estimate?

\begin{quote}
\begin{quote}
9.05\% are very similar to the 9.23\% the difference between the exact
method and the naive-bayes method is the exact method would need the the
exact same independent variable classifications to predict, where the
naive bayes method does not.
\end{quote}
\end{quote}

G. Which of the entries in this table are needed for computing P(Loan =
1 \textbar{} CC = 1, Online = 1)? Run naive Bayes on the data. Examine
the model output on training data, and find the entry that corresponds
to P(Loan = 1 \textbar{} CC = 1, Online = 1). Compare this to the number
you obtained in (E).

\begin{Shaded}
\begin{Highlighting}[]
\FunctionTok{library}\NormalTok{(}\StringTok{\textquotesingle{}e1071\textquotesingle{}}\NormalTok{)}
\end{Highlighting}
\end{Shaded}

\begin{verbatim}
## Warning: package 'e1071' was built under R version 4.0.3
\end{verbatim}

\begin{Shaded}
\begin{Highlighting}[]
\NormalTok{train }\OtherTok{=}\NormalTok{ train[,}\FunctionTok{c}\NormalTok{(}\DecValTok{10}\NormalTok{,}\DecValTok{13}\SpecialCharTok{:}\DecValTok{14}\NormalTok{)]}
\NormalTok{test }\OtherTok{=}\NormalTok{ test[,}\FunctionTok{c}\NormalTok{(}\DecValTok{10}\NormalTok{,}\DecValTok{13}\SpecialCharTok{:}\DecValTok{14}\NormalTok{)]}
\NormalTok{naivebayes }\OtherTok{=} \FunctionTok{naiveBayes}\NormalTok{(Personal.Loan}\SpecialCharTok{\textasciitilde{}}\NormalTok{.,}\AttributeTok{data=}\NormalTok{train)}
\NormalTok{naivebayes}
\end{Highlighting}
\end{Shaded}

\begin{verbatim}
## 
## Naive Bayes Classifier for Discrete Predictors
## 
## Call:
## naiveBayes.default(x = X, y = Y, laplace = laplace)
## 
## A-priori probabilities:
## Y
##          0          1 
## 0.90685714 0.09314286 
## 
## Conditional probabilities:
##    Online
## Y           0         1
##   0 0.4108381 0.5891619
##   1 0.4049080 0.5950920
## 
##    CreditCard
## Y           0         1
##   0 0.7060491 0.2939509
##   1 0.7116564 0.2883436
\end{verbatim}

\begin{quote}
\begin{quote}
The naive bayes is the exact same output we retrieved in the previous
methods. (0.288)\emph{(0.595)}(0.093)/((0.288)\emph{(0.595)}(0.093) +
(0.293)\emph{(0.589)}(0.906) = .0089 which is almost the same response
provided as above.
\end{quote}
\end{quote}

\end{document}
